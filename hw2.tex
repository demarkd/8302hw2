\documentclass[english]{article}
\newcommand{\G}{\overline{C_{2k-1}}}
\usepackage[latin9]{inputenc}
\usepackage{amsmath}
\usepackage{amssymb,mathabx}
\usepackage{lmodern}
\usepackage{mathtools}
\usepackage[inline]{enumitem}
\usepackage{relsize}
\usepackage{tikz-cd}
%\usepackage{natbib}
%\bibliographystyle{plainnat}
%\setcitestyle{authoryear,open={(},close={)}}
\let\avec=\vec
\renewcommand\vec{\mathbf}
\renewcommand{\d}[1]{\ensuremath{\operatorname{d}\!{#1}}}
\newcommand{\pydx}[2]{\frac{\partial #1}{\partial #2}}
\newcommand{\dydx}[2]{\frac{\d #1}{\d #2}}
\newcommand{\ddx}[1]{\frac{\d{}}{\d{#1}}}
\newcommand{\hk}{\hat{K}}
\newcommand{\hl}{\hat{\lambda}}
\newcommand{\ol}{\overline{\lambda}}
\newcommand{\om}{\overline{\mu}}
\newcommand{\all}{\text{all }}
\newcommand{\valph}{\vec{\alpha}}
\newcommand{\vbet}{\vec{\beta}}
\newcommand{\vT}{\vec{T}}
\newcommand{\vN}{\vec{N}}
\newcommand{\vB}{\vec{B}}
\newcommand{\vX}{\vec{X}}
\newcommand{\vx}{\vec {x}}
\newcommand{\vn}{\vec{n}}
\newcommand{\vxs}{\vec {x}^*}
\newcommand{\vV}{\vec{V}}
\newcommand{\vTa}{\vec{T}_\alpha}
\newcommand{\vNa}{\vec{N}_\alpha}
\newcommand{\vBa}{\vec{B}_\alpha}
\newcommand{\vTb}{\vec{T}_\beta}
\newcommand{\vNb}{\vec{N}_\beta}
\newcommand{\vBb}{\vec{B}_\beta}
\newcommand{\bvT}{\bar{\vT}}
\newcommand{\ka}{\kappa_\alpha}
\newcommand{\ta}{\tau_\alpha}
\newcommand{\kb}{\kappa_\beta}
\newcommand{\tb}{\tau_\beta}
\newcommand{\hth}{\hat{\theta}}
\newcommand{\evat}[3]{\left. #1\right|_{#2}^{#3}}
\newcommand{\restr}[2]{\evat{#1}{#2}{}}
\newcommand{\prompt}[1]{\begin{prompt*}
		#1
\end{prompt*}}
\newcommand{\vy}{\vec{y}}
\DeclareMathOperator{\sech}{sech}
\DeclarePairedDelimiter\abs{\lvert}{\rvert}%
\DeclarePairedDelimiter\norm{\lVert}{\rVert}%
\newcommand{\dis}[1]{\begin{align}
	#1
	\end{align}}
\newcommand{\LL}{\mathcal{L}}
\newcommand{\RR}{\mathbb{R}}
\newcommand{\CC}{\mathbb{C}}
\newcommand{\NN}{\mathbb{N}}
\newcommand{\ZZ}{\mathbb{Z}}
\newcommand{\QQ}{\mathbb{Q}}
\newcommand{\Ss}{\mathcal{S}}
\newcommand{\BB}{\mathcal{B}}
\usepackage{graphicx}
% Swap the definition of \abs* and \norm*, so that \abs
% and \norm resizes the size of the brackets, and the 
% starred version does not.
%\makeatletter
%\let\oldabs\abs
%\def\abs{\@ifstar{\oldabs}{\oldabs*}}
%
%\let\oldnorm\norm
%\def\norm{\@ifstar{\oldnorm}{\oldnorm*}}
%\makeatother
\newenvironment{subproof}[1][\proofname]{%
	\renewcommand{\qedsymbol}{$\blacksquare$}%
	\begin{proof}[#1]%
	}{%
	\end{proof}%
}

\usepackage{centernot}
\usepackage{dirtytalk}
\usepackage{calc}
\newcommand{\prob}[1]{\setcounter{section}{#1-1}\section{}}


\newcommand{\prt}[1]{\setcounter{subsection}{#1-1}\subsection{}}
\newcommand{\pprt}[1]{{\textit{{#1}.)}}\newline}
\renewcommand\thesubsection{\alph{subsection}}
\usepackage[sl,bf,compact]{titlesec}
\titlelabel{\thetitle.)\quad}
\DeclarePairedDelimiter\floor{\lfloor}{\rfloor}
\makeatletter

\newcommand*\pFqskip{8mu}
\catcode`,\active
\newcommand*\pFq{\begingroup
	\catcode`\,\active
	\def ,{\mskip\pFqskip\relax}%
	\dopFq
}
\catcode`\,12
\def\dopFq#1#2#3#4#5{%
	{}_{#1}F_{#2}\biggl(\genfrac..{0pt}{}{#3}{#4}|#5\biggr
	)%
	\endgroup
}
\def\res{\mathop{Res}\limits}
% Symbols \wedge and \vee from mathabx
% \DeclareFontFamily{U}{matha}{\hyphenchar\font45}
% \DeclareFontShape{U}{matha}{m}{n}{
%       <5> <6> <7> <8> <9> <10> gen * matha
%       <10.95> matha10 <12> <14.4> <17.28> <20.74> <24.88> matha12
%       }{}
% \DeclareSymbolFont{matha}{U}{matha}{m}{n}
% \DeclareMathSymbol{\wedge}         {2}{matha}{"5E}
% \DeclareMathSymbol{\vee}           {2}{matha}{"5F}
% \makeatother

%\titlelabel{(\thesubsection)}
%\titlelabel{(\thesubsection)\quad}
\usepackage{listings}
\lstloadlanguages{[5.2]Mathematica}
\usepackage{babel}
\newcommand{\ffac}[2]{{(#1)}^{\underline{#2}}}
\usepackage{color}
\usepackage{amsthm}
\newtheorem{theorem}{Theorem}[section]
\newtheorem*{theorem*}{Theorem}
\newtheorem{conj}[theorem]{Conjecture}
\newtheorem{corollary}[theorem]{Corollary}
\newtheorem{example}[theorem]{Example}
\newtheorem{lemma}[theorem]{Lemma}
\newtheorem*{lemma*}{Lemma}
\newtheorem{problem}[theorem]{Problem}
\newtheorem{proposition}[theorem]{Proposition}
\newtheorem*{proposition*}{Proposition}
\newtheorem*{corollary*}{Corollary}
\newtheorem{fact}[theorem]{Fact}
\newtheorem*{prompt*}{Prompt}
\newtheorem*{claim*}{Claim}
\newtheorem{claim}{Claim}
%\newcommand{\claim}[1]{\begin{claim*} #1\end{claim*}}
%organizing theorem environments by style--by the way, should we really have definitions (and notations I guess) in proposition style? it makes SO much of our text italicized, which is weird.
\theoremstyle{remark}
\newtheorem{remark}{Remark}[section]

\theoremstyle{definition}
\newtheorem{definition}[theorem]{Definition}
\newtheorem*{definition*}{Definition}
\newtheorem{notation}[theorem]{Notation}
\newtheorem*{notation*}{Notation}
%FINAL
\newcommand{\due}{26 February 2017} 
\RequirePackage{geometry}
\geometry{margin=.7in}
\usepackage{todonotes}
\title{MATH 8302 Homework II}
\author{David DeMark}
\date{\due}
\usepackage{fancyhdr}
\pagestyle{fancy}
\fancyhf{}
\rhead{David DeMark}
\chead{\due}
\lhead{MATH 8302}
\cfoot{\thepage}
% %%
%%
%%
%DRAFT

%\usepackage[left=1cm,right=4.5cm,top=2cm,bottom=1.5cm,marginparwidth=4cm]{geometry}
%\usepackage{todonotes}
% \title{MATH 8669 Homework 4-DRAFT}
% \usepackage{fancyhdr}
% \pagestyle{fancy}
% \fancyhf{}
% \rhead{David DeMark}
% \lhead{MATH 8669-Homework 4-DRAFT}
% \cfoot{\thepage}

%PROBLEM SPEFICIC

\newcommand{\lint}{\underline{\int}}
\newcommand{\uint}{\overline{\int}}
\newcommand{\hfi}{\hat{f}^{-1}}
\newcommand{\tfi}{\tilde{f}^{-1}}
\newcommand{\tsi}{\tilde{f}^{-1}}
\newcommand{\PP}{\mathcal{P}}
\newcommand{\nin}{\centernot\in}
\newcommand{\seq}[1]{({#1}_n)_{n\geq 1}}
\newcommand{\Tt}{\mathcal{T}}
\newcommand{\card}{\mathrm{card}}
\newcommand{\setc}[2]{\{ #1\::\:#2 \}}
\newcommand{\Fcal}{\mathcal{F}}
\newcommand{\cbal}{\overline{B}}
\newcommand{\Ccal}{\mathcal{C}}
\newcommand{\Dcal}{\mathcal{D}}
\newcommand{\cl}{\overline}
\newcommand{\id}{\mathrm{id}}
\newcommand{\intr}{\mathrm{int}}
\renewcommand{\hom}{\mathrm{Hom}}
\newcommand{\vect}{\mathrm{Vect}}
\newcommand{\Top}{\mathrm{Top}}
\renewcommand{\top}{\Top}
\newcommand{\hTop}{\mathrm{hTop}}
\newcommand{\set}{\mathrm{Set}}
\newcommand{\frp}{\mathop{\large {\mathlarger{*}}}}
\newcommand{\ondt}{1_{\cdot}}
\newcommand{\onst}{1_{\star}}
\newcommand{\bdy}{\partial}
\newcommand{\im}{\mathrm{im}}
\newcommand{\re}{\mathrm{re}}
\newcommand{\oX}{\overline{X}}
\newcommand{\ox}{\overline{x}}
\newcommand{\tX}{\tilde{X}}
\newcommand{\tH}{\tilde{H}}
\newcommand{\tx}{\tilde{x}}
\newcommand{\hX}{\hat{X}}
\newcommand{\hx}{\hat{x}}
\newcommand{\aut}{\mathrm{Aut}}
\newcommand{\del}{\partial}
\newcommand{\RP}{{\RR\mathrm{P}}}
\newcommand{\CP}{{\CC\mathrm{P}}}
\newcommand{\csm}{\RP^n\#\RP^n}
\DeclareMathOperator{\coker}{coker}
\newcommand{\idl}[1]{\langle #1\rangle}
\renewcommand{\thetheorem}{\arabic{section}.\Alph{theorem}}
\DeclareMathOperator{\ext}{Ext}
\begin{document}
\maketitle
\prob{1}\prt{1}\begin{proposition*}
	We let $f:\CP^n\to \CP^n$ be a continuous map with the property that $f^*:H^2(\CP^n,\ZZ)\to H^2(\CP^n,\ZZ)\cong \ZZ$ is the multiplication-by-$d$ map for some $d\in \ZZ$. Then, $f^*:H^k(\CP^n,\ZZ)\to H^k(\CP^n,\ZZ)$ is the multiplication-by-$d^m$ map for $k=2m\leq 2n$ and the zero map otherwise.
\end{proposition*}
\begin{proof}
 We have that under the ring structure endowed upon $H^*(\CP^n,\ZZ)$ by addition and the cup product, $H^*(\CP^n,\ZZ)\cong \ZZ[c]/\idl{c^{n+1}}$ where $\deg c=2$. Thus, $H^2(\CP^n)=\ZZ\{c\}$, so the map $f^*:H^2(\CP^n)\to H^2 (\CP^n)$ maps $c\mapsto d*c$. Then, the map $f^*$ on $H^*(\CP^n)$ maps $c^m\mapsto (dc)^m=d^mc^m$ and hence induces the multiplication-by-$d^m$ on $H^{2m}(\CP^n)=\ZZ\{c^m\}$. On the other hand, if $k$ is odd or $k>2n$, then $H^k(\CP^n)=0$, so necessarily, $f^*$ is the zero map.
\end{proof}

\prt{2}
\begin{lemma}\label{1bcoin}
	$f_*:H_{k}(\CP^n,\ZZ)\to H_k(\CP^n,\ZZ)$ is the multiplication-by-$d^m$ map for $0\leq k=2m\leq 2n$ and the zero map otherwise.
\end{lemma}
\begin{subproof}[Proof of Lemma \ref{1bcoin}]
We first note that, as before, if $k$ is odd, $k<0$ or $k>2n$, then $H_k(\CP^n)=0$ so $f_*$ is, necessarily, the zero map. We now assume $k=2m$. We note by naturality of the Universal Coefficient Theorem, we have the commuting diagram of \eqref{1bUCT} with exactness along the rows.
\begin{equation}\label{1bUCT}
	\begin{tikzcd}
	0\arrow[r]&\ext(H_{k-1}(\CP^n),\ZZ)\arrow[r]&H^k(\CP^n,\ZZ)\arrow[r,"h"]&\hom(H_k(\CP^n),\ZZ)\arrow[r]&0\\
	0\arrow[r]&\ext(H_{k-1}(\CP^n),\ZZ)\arrow[r]\arrow[u,"(f_*)^*"]&H^k(\CP^n,\ZZ)\arrow[r,"h"]\arrow[u,"f^*"]&\hom(H_k(\CP^n),\ZZ)\arrow[r]\arrow[u,"(f_*)^*"]&0
	\end{tikzcd}
\end{equation}

As $k$ is even by assumption, we have that $H_{k-1}(\CP^n)=0$. Hence, $\ext(H_{k-1}(\CP^n),\ZZ)=0$, and we are left with the diagram of \eqref{1bUCT'}
\begin{equation}\label{1bUCT'}
\begin{tikzcd}
0\arrow[r]&H^k(\CP^n,\ZZ)\arrow[r,"h"]&\hom(H_k(\CP^n),\ZZ)\arrow[r]&0\\
0\arrow[r]&H^k(\CP^n,\ZZ)\arrow[r,"h"]\arrow[u,"f^*"]&\hom(H_k(\CP^n),\ZZ)\arrow[r]\arrow[u,"(f_*)^*"]&0
\end{tikzcd}
\end{equation}
Thus, in particular $h:H^k(\CP^n,\ZZ)\cong \ZZ\to \hom(H_k(\CP^n),\ZZ)\cong \ZZ$ is an isomorphism and hence the multiplication-by-$\pm1$ map. In either case (that is, $-1$ or $+1$), a very simple diagram-chase shows that $(f_*)^*$ is necessarily the multiplication-by-$d^m$ map. We fix a generator $g\in H_k(\CP^n)$ and let $\hom(H_k(\CP^n),\ZZ)=\{\phi_\ell\}_{\ell\in\ZZ}$ where $\phi_\ell:g\mapsto \ell $. We fix the isomorphism to $\ZZ$ mapping $\phi_1\to 1$ as an identification. We have that $(f_*)^*:\phi_1\mapsto \phi_1\circ f_*$, and as this is the multiplication-by-$d^m$ map, we have that $\phi_1\circ f_*=\phi_{d^m}$. As $\phi_1$ is the identity map on $\ZZ$ under the identification $H_k(\CP^n)\cong \ZZ$ by $g\mapsto 1$ (which is chosen to be compatible with our identification $ \hom(H_k(\CP^n),\ZZ)\cong \ZZ$, we have that indeed $f_*:g\to d^m g$ and is hence the multiplication-by-$d^m$ map.
\end{subproof}
\begin{proposition*}[Main proposition of problem]
	$\tau(f)=\sum_{m=0}^nd^m$. In particular, $f$ has a fixed point unless $n\equiv 1 \mod 2$ and the induced map of $f^2$ on homology is the identity.  
\end{proposition*}
\begin{proof}
	We note that by Lemma \ref{1bcoin}, \begin{equation*}\mathrm{tr}(f_*:H_k(\CP^n)\to H_k(\CP^n))=\begin{cases}
d^m&0\leq k=2m\leq 2n\\0&\text{else}.
	\end{cases}\end{equation*}
\end{proof}
Thus, the odd terms in the summation for $\tau(f)$ uniformly vanish and we are left to sum the even terms. We note that the function $F(d)=1+d+\ldots+d^n$ divides $1-d^{n+1}$ and hence has zeros all (possibly non-primitive) $(n+1)$-st roots of unity not including $1$. Thus, $\tau(f)=0$ (and thus $f$ is fixed point free) only when $(n+1)$ is odd and $f_*:H_k(\CP^n)\to H_k(\CP^n)$ is the multiplication-by-$(-1)$ map. 
\prt{3}
\begin{proposition*}
	There are no orientation-reversing maps of $\CP^n$ for even $n$.
\end{proposition*}
\begin{proof}
We say $f$ is orientation-reversing on $\CP^n$ if $f^*$ is multiplication by a negative number on $H_{2n}(\CP^n)$. As we have already seen, $f^*$ is the multiplication-by-$d^n$ map on $H_{2n}(\CP^n)$. Thus, if $n$ is even, $d^n$ is necessarily nonnegative and $f$ is not orientation-reversing.
\end{proof}
\prob{2}
\prob{3}
\begin{proposition*}
	We let $p:\RR^k\to \RR$ be a homogeneous degree-$m$ polynomial, and $V_a=p^{-1}(a)$ where $0\neq a\in \RR$. Then, $V_a$ is a submanifold of $\RR^k$ of dimension $k-1$ and for $a\neq b$ with $ab>0$, $V_a$ and $V_b$ are diffeomorphic.
\end{proposition*}
\begin{proof}
	We note that if $a$ is a regular value for $p$, the pre-image theorem asserts that $V_a$ is a submanifold of dimension $k-1$ indeed. We thus wish to show that $a$ is a regular value for $p$. By Euler's identity for homogeneous polynomials, we have that for $x\in V_a$,
	$$\sum_{j=1}^kx_i\pydx{p}{x_i}(x)=mp(x)=ma$$
	Thus, we have that for some $i$ $\pydx{p}{x_i}(x)\neq 0$, so $dp_x$ is indeed surjective as its target is one-dimensional.
	
	Now, to see that $V_a$ and $V_b$ are related by diffeomorphism, we let $f_t:\RR^k\to \RR^k$ be the isomorphism scaling by $t=\left(\frac{a}{b}\right)^{1/m}$. As $\frac{a}{b}>0$, such a $t$ certainly exists in the reals, and it is clear that multiplication by a scalar is a diffeomorphism on real space. Furthermore, we see that for $x\in V_b$, $p(f_t(x))=p(tx)=t^dp(x)=\frac{a}{b}*b=a$, so $f_t(x)\in V_a$; an identical argument shows that $f_t^{-1}$ maps $V_a$ to $V_b$. As the embeddings $V_a\hookleftarrow \RR^k$, $V_b\hookleftarrow \RR^k$ are diffeomorphisms onto their images, we now have that $f_t:V_b\to V_a$ is the composition of diffeomorphisms and hence itself a diffeomorphism.  
	
\end{proof}

\end{document}
